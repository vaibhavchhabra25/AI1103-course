\documentclass[journal,12pt,twocolumn]{IEEEtran}

\usepackage{setspace}
\usepackage{gensymb}
\singlespacing
\usepackage[cmex10]{amsmath}

\usepackage{amsthm}

\usepackage{mathrsfs}
\usepackage{txfonts}
\usepackage{stfloats}
\usepackage{bm}
\usepackage{cite}
\usepackage{cases}
\usepackage{subfig}

\usepackage{longtable}
\usepackage{multirow}

\usepackage{enumitem}
\usepackage{mathtools}
\usepackage{steinmetz}
\usepackage{tikz}
\usepackage{circuitikz}
\usepackage{verbatim}
\usepackage{tfrupee}
\usepackage[breaklinks=true]{hyperref}
\usepackage{graphicx}
\usepackage{tkz-euclide}

\usetikzlibrary{calc,math}
\usepackage{listings}
    \usepackage{color}                                            %%
    \usepackage{array}                                            %%
    \usepackage{longtable}                                        %%
    \usepackage{calc}                                             %%
    \usepackage{multirow}                                         %%
    \usepackage{hhline}                                           %%
    \usepackage{ifthen}                                           %%
    \usepackage{lscape}     
\usepackage{multicol}
\usepackage{chngcntr}

\DeclareMathOperator*{\Res}{Res}

\renewcommand\thesection{\arabic{section}}
\renewcommand\thesubsection{\thesection.\arabic{subsection}}
\renewcommand\thesubsubsection{\thesubsection.\arabic{subsubsection}}

\renewcommand\thesectiondis{\arabic{section}}
\renewcommand\thesubsectiondis{\thesectiondis.\arabic{subsection}}
\renewcommand\thesubsubsectiondis{\thesubsectiondis.\arabic{subsubsection}}


\hyphenation{op-tical net-works semi-conduc-tor}
\def\inputGnumericTable{}                                 %%

\lstset{
%language=C,
frame=single, 
breaklines=true,
columns=fullflexible
}
\begin{document}


\newtheorem{theorem}{Theorem}[section]
\newtheorem{problem}{Problem}
\newtheorem{proposition}{Proposition}[section]
\newtheorem{lemma}{Lemma}[section]
\newtheorem{corollary}[theorem]{Corollary}
\newtheorem{example}{Example}[section]
\newtheorem{definition}[problem]{Definition}

\newcommand{\BEQA}{\begin{eqnarray}}
\newcommand{\EEQA}{\end{eqnarray}}
\newcommand{\define}{\stackrel{\triangle}{=}}
\bibliographystyle{IEEEtran}
\raggedbottom
\setlength{\parindent}{0pt}
\providecommand{\mbf}{\mathbf}
\providecommand{\pr}[1]{\ensuremath{\Pr\left(#1\right)}}
\providecommand{\qfunc}[1]{\ensuremath{Q\left(#1\right)}}
\providecommand{\sbrak}[1]{\ensuremath{{}\left[#1\right]}}
\providecommand{\lsbrak}[1]{\ensuremath{{}\left[#1\right.}}
\providecommand{\rsbrak}[1]{\ensuremath{{}\left.#1\right]}}
\providecommand{\brak}[1]{\ensuremath{\left(#1\right)}}
\providecommand{\lbrak}[1]{\ensuremath{\left(#1\right.}}
\providecommand{\rbrak}[1]{\ensuremath{\left.#1\right)}}
\providecommand{\cbrak}[1]{\ensuremath{\left\{#1\right\}}}
\providecommand{\lcbrak}[1]{\ensuremath{\left\{#1\right.}}
\providecommand{\rcbrak}[1]{\ensuremath{\left.#1\right\}}}
\theoremstyle{remark}
\newtheorem{rem}{Remark}
\newcommand{\sgn}{\mathop{\mathrm{sgn}}}
\providecommand{\abs}[1]{\left\vert#1\right\vert}
\providecommand{\res}[1]{\Res\displaylimits_{#1}} 
\providecommand{\norm}[1]{\left\lVert#1\right\rVert}
%\providecommand{\norm}[1]{\lVert#1\rVert}
\providecommand{\mtx}[1]{\mathbf{#1}}
\providecommand{\mean}[1]{E\left[ #1 \right]}
\providecommand{\fourier}{\overset{\mathcal{F}}{ \rightleftharpoons}}
%\providecommand{\hilbert}{\overset{\mathcal{H}}{ \rightleftharpoons}}
\providecommand{\system}{\overset{\mathcal{H}}{ \longleftrightarrow}}
	%\newcommand{\solution}[2]{\textbf{Solution:}{#1}}
\newcommand{\solution}{\noindent \textbf{Solution: }}
\newcommand{\cosec}{\,\text{cosec}\,}
\providecommand{\dec}[2]{\ensuremath{\overset{#1}{\underset{#2}{\gtrless}}}}
\newcommand{\myvec}[1]{\ensuremath{\begin{pmatrix}#1\end{pmatrix}}}
\newcommand{\mydet}[1]{\ensuremath{}}
\numberwithin{equation}{subsection}
\makeatletter
\@addtoreset{figure}{problem}
\makeatother
\let\StandardTheFigure\thefigure
\let\vec\mathbf
\renewcommand{\thefigure}{\theproblem}
\def\putbox#1#2#3{\makebox[0in][l]{\makebox[#1][l]{}\raisebox{\baselineskip}[0in][0in]{\raisebox{#2}[0in][0in]{#3}}}}
     \def\rightbox#1{\makebox[0in][r]{#1}}
     \def\centbox#1{\makebox[0in]{#1}}
     \def\topbox#1{\raisebox{-\baselineskip}[0in][0in]{#1}}
     \def\midbox#1{\raisebox{-0.5\baselineskip}[0in][0in]{#1}}
\vspace{3cm}
\title{Assignment 3}
\author{Vaibhav Chhabra \\ AI20BTECH11022}
\maketitle
\newpage
\bigskip
\renewcommand{\thefigure}{\theenumi}
\renewcommand{\thetable}{\theenumi}
Download all latex codes from 
\begin{lstlisting}
    https://github.com/vaibhavchhabra25/AI1103-course/blob/main/Assignment-3/main.tex
\end{lstlisting}
\section{Problem}
(GATE 2001 (MA), Q. 2.24)
Let $(X,Y)$ be a two-dimensional random variable such that $E(X)=E(Y)=3$, $Var(X)=Var(Y)=1$ and $Cov(X,Y)=1/2$.
Then, $P(|X-Y|>6)$ is
\begin{enumerate}
\begin{multicols}{2}
    \item less than 1/6
    \item equal to 1/2
    \item equal to 1/3
    \item greater than 1/2
\end{multicols}
\end{enumerate}

\section{Solution}
Given,
\begin{align} \label{eq-1}
    E(X)=E(Y)=3
\end{align}
\begin{align}
    Var(X)=Var(Y)=1
\end{align}
\begin{align}
    Cov(X,Y)=1/2
\end{align}
Now,
\begin{align}
    Var(X)=E(X^2)-(E(X))^2
\end{align}
Substituting given values, we get,
\begin{align}
    1=E(X^2)-3^2
\end{align}
So,
\begin{align} \label{eq-2}
    E(X^2)=10
\end{align}
Similarly for $Y$,
\begin{align} \label{eq-3}
    E(Y^2)=10
\end{align}
Also,
\begin{align}
    Cov(X,Y)=E(XY)-E(X)E(Y)
\end{align}
Substituting given values, we get,
\begin{align}
    1/2=E(XY)-(3)(3)
\end{align}
So,
\begin{align} \label{eq-4}
    E(XY)=19/2
\end{align}

Let $Z$ be a random variable defined as
\begin{align} \label{eq-5}
    Z=X-Y
\end{align}
Then using \eqref{eq-1},
\begin{align} \label{eq-6}
    E(Z)=E(X-Y)=E(X)-E(Y)=0
\end{align}
Now, using \eqref{eq-6}
\begin{align}
    Var(Z)=E(Z^2)-(E(Z))^2=E(Z^2)
\end{align}
\begin{align}
    Var(Z)=E((X-Y)^2)
\end{align}
\begin{align}
    Var(Z)=E(X^2)+E(Y^2)-2E(XY)
\end{align}
Using \eqref{eq-2}, \eqref{eq-3} and \eqref{eq-4},
\begin{align}
    Var(Z)=10+10-2 \times 19/2
\end{align}
\begin{align} \label{eq-7}
    Var(Z)=1
\end{align}

\begin{theorem}
(Chebychev's Inequality)\ Let T be an arbitrary random variable, with finite mean E(T), then for all $a>0$,
\begin{align}
    \pr{|T-E(T)| \geq a} \leq \dfrac{Var(T)}{a^2}
\end{align}
\end{theorem}

\begin{proof}
Let T be a random variable with probablility distribution function $f(T)$ and $a>0$ be any real number.
Then,
\begin{align}
    \pr{|T-E(T)| \geq a}= \int_{-\infty}^{-E(T)-a}f(T)dT + \nonumber \\  \int_{E(T)+a}^{\infty}f(T)dT
\end{align}
\begin{align} \label{eq-8}
    \pr{|T-E(T)| \geq a}= \int_{|T-E(T)| \geq a}f(T)dT
\end{align}
Now,
\begin{align}
    Var(T)= \int_{-\infty}^{\infty}(T-E(T))^2f(T)dT \nonumber \\ \geq \int_{|T-E(T)|\geq a}(T-E(T))^2f(T)dT \nonumber \\ \geq a^2\int_{|T-E(T)|\geq a}f(T)dT
\end{align}
So, we finally get,
\begin{align}
    Var(T) \geq a^2\int_{|T-E(T)|\geq a}f(T)dT
\end{align}
Using \eqref{eq-8},
\begin{align}
    Var(T) \geq a^2\pr{|T-E(T)| \geq a}
\end{align}
Or,
\begin{align}
    \pr{|T-E(T)| \geq a} \leq \dfrac{Var(T)}{a^2}
\end{align}
\end{proof}
Applying Chebychev's Inequality for $Z$ with $a=6$, we get,
\begin{align}
    \pr{|Z-E(Z)| \geq 6} \leq \dfrac{Var(Z)}{6^2}
\end{align}
Using \eqref{eq-6} and \eqref{eq-7},
\begin{align}
    \pr{|Z-0| \geq 6} \leq \dfrac{1}{36}
\end{align}
As $Z=X-Y$,
\begin{align}
    \pr{|X-Y| \geq 6} \leq \dfrac{1}{36}
\end{align}
So, option 1 is correct.
\end{document}
