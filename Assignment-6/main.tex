\documentclass[journal,12pt,twocolumn]{IEEEtran}

\usepackage{setspace}
\usepackage{gensymb}
\singlespacing
\usepackage[cmex10]{amsmath}

\usepackage{amsthm}

\usepackage{mathrsfs}
\usepackage{txfonts}
\usepackage{stfloats}
\usepackage{bm}
\usepackage{cite}
\usepackage{cases}
\usepackage{subfig}

\usepackage{longtable}
\usepackage{multirow}

\usepackage{enumitem}
\usepackage{mathtools}
\usepackage{steinmetz}
\usepackage{tikz}
\usepackage{circuitikz}
\usepackage{verbatim}
\usepackage{tfrupee}
\usepackage[breaklinks=true]{hyperref}
\usepackage{graphicx}
\usepackage{tkz-euclide}

\usetikzlibrary{calc,math}
\usepackage{listings}
    \usepackage{color}                                            %%
    \usepackage{array}                                            %%
    \usepackage{longtable}                                        %%
    \usepackage{calc}                                             %%
    \usepackage{multirow}                                         %%
    \usepackage{hhline}                                           %%
    \usepackage{ifthen}                                           %%
    \usepackage{lscape}     
\usepackage{multicol}
\usepackage{chngcntr}

\DeclareMathOperator*{\Res}{Res}

\renewcommand\thesection{\arabic{section}}
\renewcommand\thesubsection{\thesection.\arabic{subsection}}
\renewcommand\thesubsubsection{\thesubsection.\arabic{subsubsection}}

\renewcommand\thesectiondis{\arabic{section}}
\renewcommand\thesubsectiondis{\thesectiondis.\arabic{subsection}}
\renewcommand\thesubsubsectiondis{\thesubsectiondis.\arabic{subsubsection}}


\hyphenation{op-tical net-works semi-conduc-tor}
\def\inputGnumericTable{}                                 %%

\lstset{
%language=C,
frame=single, 
breaklines=true,
columns=fullflexible
}
\begin{document}


\newtheorem{theorem}{Theorem}[section]
\newtheorem{problem}{Problem}
\newtheorem{proposition}{Proposition}[section]
\newtheorem{lemma}{Lemma}[section]
\newtheorem{corollary}[theorem]{Corollary}
\newtheorem{example}{Example}[section]
\newtheorem{definition}[problem]{Definition}

\newcommand{\BEQA}{\begin{eqnarray}}
\newcommand{\EEQA}{\end{eqnarray}}
\newcommand{\define}{\stackrel{\triangle}{=}}
\bibliographystyle{IEEEtran}
\raggedbottom
\setlength{\parindent}{0pt}
\providecommand{\mbf}{\mathbf}
\providecommand{\pr}[1]{\ensuremath{\Pr\left(#1\right)}}
\providecommand{\qfunc}[1]{\ensuremath{Q\left(#1\right)}}
\providecommand{\sbrak}[1]{\ensuremath{{}\left[#1\right]}}
\providecommand{\lsbrak}[1]{\ensuremath{{}\left[#1\right.}}
\providecommand{\rsbrak}[1]{\ensuremath{{}\left.#1\right]}}
\providecommand{\brak}[1]{\ensuremath{\left(#1\right)}}
\providecommand{\lbrak}[1]{\ensuremath{\left(#1\right.}}
\providecommand{\rbrak}[1]{\ensuremath{\left.#1\right)}}
\providecommand{\cbrak}[1]{\ensuremath{\left\{#1\right\}}}
\providecommand{\lcbrak}[1]{\ensuremath{\left\{#1\right.}}
\providecommand{\rcbrak}[1]{\ensuremath{\left.#1\right\}}}
\theoremstyle{remark}
\newtheorem{rem}{Remark}
\newcommand{\sgn}{\mathop{\mathrm{sgn}}}
\providecommand{\abs}[1]{\left\vert#1\right\vert}
\providecommand{\res}[1]{\Res\displaylimits_{#1}} 
\providecommand{\norm}[1]{\left\lVert#1\right\rVert}
%\providecommand{\norm}[1]{\lVert#1\rVert}
\providecommand{\mtx}[1]{\mathbf{#1}}
\providecommand{\mean}[1]{E\left[ #1 \right]}
\providecommand{\fourier}{\overset{\mathcal{F}}{ \rightleftharpoons}}
%\providecommand{\hilbert}{\overset{\mathcal{H}}{ \rightleftharpoons}}
\providecommand{\system}{\overset{\mathcal{H}}{ \longleftrightarrow}}
	%\newcommand{\solution}[2]{\textbf{Solution:}{#1}}
\newcommand{\solution}{\noindent \textbf{Solution: }}
\newcommand{\cosec}{\,\text{cosec}\,}
\providecommand{\dec}[2]{\ensuremath{\overset{#1}{\underset{#2}{\gtrless}}}}
\newcommand{\myvec}[1]{\ensuremath{\begin{pmatrix}#1\end{pmatrix}}}
\newcommand{\mydet}[1]{\ensuremath{}}
\numberwithin{equation}{subsection}
\makeatletter
\@addtoreset{figure}{problem}
\makeatother
\let\StandardTheFigure\thefigure
\let\vec\mathbf
\renewcommand{\thefigure}{\theproblem}
\def\putbox#1#2#3{\makebox[0in][l]{\makebox[#1][l]{}\raisebox{\baselineskip}[0in][0in]{\raisebox{#2}[0in][0in]{#3}}}}
     \def\rightbox#1{\makebox[0in][r]{#1}}
     \def\centbox#1{\makebox[0in]{#1}}
     \def\topbox#1{\raisebox{-\baselineskip}[0in][0in]{#1}}
     \def\midbox#1{\raisebox{-0.5\baselineskip}[0in][0in]{#1}}
\vspace{3cm}
\title{Assignment 6}
\author{Vaibhav Chhabra \\ AI20BTECH11022}
\maketitle
\newpage
\bigskip
\renewcommand{\thefigure}{\theenumi}
\renewcommand{\thetable}{\theenumi}
Download all latex codes from 
\begin{lstlisting}
    https://github.com/vaibhavchhabra25/AI1103-course/blob/main/Assignment-6/main.tex
\end{lstlisting}
\section{Problem}
(CSIR UGC NET EXAM June 2013 - Q.75)\\
Let $X$ be a non-negative integer valued random variable with probability mass function $f(x)$ satisfying $(x+1)f(x+1)=(\alpha + \beta x)f(x)$, $x=0,1,2,...$; $\beta \neq 1$. You may assume that $E(X)$ and $Var(X)$ exist. Then which of the following statements are true?
\vspace{0.2cm}
\begin{enumerate}
    \item $E(X)=\dfrac{\alpha}{1-\beta}$ \vspace{0.2cm}
    \item $E(X)=\dfrac{\alpha^2}{(1-\beta)(1+\alpha)}$ \vspace{0.2cm}
    \item $Var(X)=\dfrac{\alpha^2}{(1-\beta)^2}$ \vspace{0.2cm}
    \item $Var(X)=\dfrac{\alpha}{(1-\beta)^2}$
\end{enumerate}

\section{Solution}
For a discrete random variable $X$ with P.D.F. $f(x)$ and which can take values from a set $\mathbb{S}$,
\begin{align} \label{eq-1}
    E(X)= \sum_{x \in \mathbb{S}}xf(x)
\end{align}
And,
\begin{align} \label{eq-2}
    E(X^2) =\sum_{x \in \mathbb{S}}x^2f(x)
\end{align}
Also, as $f(x)$ is the P.D.F.,
\begin{align} \label{eq-3}
    \sum_{x \in \mathbb{S}}f(x) = 1
\end{align}
Given, for $x \in \mathbb{S}=\{0,1,2,...n\}$,
\begin{align} \label{eq-4}
    (x+1)f(x+1)=(\alpha + \beta x)f(x)
\end{align}
Summing both sides for $x \in \mathbb{S}$ we get,
\begin{align}
    \sum_{x=0}^n(x+1)f(x+1)=\sum_{x=0}^n(\alpha +\beta x)f(x)
\end{align}
Replacing $x+1$ with $x$ in L.H.S. we get, 
\begin{align}
    \sum_{x=1}^{n+1}xf(x)=\sum_{x=0}^n(\alpha +\beta x)f(x)
\end{align}
Rewriting LHS, we get,
\begin{align}
    \sum_{x=0}^nxf(x)+(n+1)f(n+1)=\sum_{x=0}^n(\alpha +\beta x)f(x)
\end{align}
But as $x \in \{0,1,2...n\}$, $f(n+1)=0$. So the equation becomes
\begin{align}
    \sum_{x=0}^nxf(x)=\alpha \sum_{x=0}^nf(x) + \beta \sum_{x=0}^nxf(x)
\end{align}
Using \eqref{eq-1} and \eqref{eq-3}, we get,
\begin{align} 
    E(X)=\alpha(1) + \beta E(X)
\end{align}
So,
\begin{align} \label{eq-5}
    E(X)=\dfrac{\alpha}{1-\beta}
\end{align}
Now in \eqref{eq-4}, multiplying both sides by $(x+1)$, we get,
\begin{align}
    (x+1)^2f(x+1)=(\alpha + \beta x)(x+1)f(x)
\end{align}
Summing both sides for $x \in \mathbb{S}$ we get,
\begin{align}
    \sum_{x=0}^n(x+1)^2f(x+1)=\sum_{x=0}^n(\alpha +\beta x)(x+1)f(x)
\end{align}
Replacing $x+1$ with $x$ in L.H.S. we get, 
\begin{align}
    \sum_{x=1}^{n+1}x^2f(x)=\sum_{x=0}^n(\beta x^2f(x) + (\alpha+\beta)xf(x) + \alpha f(x))
\end{align}
Rewriting LHS similarly as before, we get,
\begin{align}
    \sum_{x=0}^nx^2f(x)=\beta \sum_{x=0}^nx^2f(x) + \nonumber \\
    (\alpha + \beta)\sum_{x=0}^nxf(x) + \alpha \sum_{x=0}^nf(x)
\end{align}
Using \eqref{eq-1}, \eqref{eq-2} and \eqref{eq-3}, we get,
\begin{align}
    E(X^2)=\beta E(X^2) + (\alpha + \beta)E(X) + \alpha (1) 
\end{align}
Using \eqref{eq-5}
\begin{align}
    E(X^2)(1-\beta)=\dfrac{\alpha(\alpha+\beta)}{1-\beta} + \alpha
\end{align}
So,
\begin{align} \label{eq-6}
    E(X^2)=\dfrac{\alpha^2+\alpha}{(1-\beta)^2}
\end{align}
Now,
\begin{align}
    Var(X)=E(X^2)-(E(X))^2
\end{align}
Using \eqref{eq-5} and \eqref{eq-6},
\begin{align}
    Var(X)=\dfrac{\alpha^2+\alpha}{(1-\beta)^2}-\dfrac{\alpha^2}{(1-\beta)^2}
\end{align}
So,
\begin{align}
    Var(X)=\dfrac{\alpha}{(1-\beta)^2}
\end{align}
So, options 1 and 4 are correct.
\end{document}
